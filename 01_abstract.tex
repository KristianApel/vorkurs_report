\chapter*{Abstract}
The structural mechanisms by which receptor tyrosine kinases (RTKs) regulate catalytic activity are diverse and often based on subtle changes in conformational dynamics. The regulatory mechanism of one such RTK, fibroblast growth factor receptor 2 (FGFR2) kinase, is still unknown, as the numerous crystal structures of the unphosphorylated and phosphorylated forms of the kinase domains show no apparent structural change that could explain how phosphorylation could enable catalytic activity. In this study, we use several enhanced sampling molecular dynamics (MD) methods to elucidate the structural changes to the kinase’s activation loop that occur upon phosphorylation. We show that phosphorylation favors inward motion of Arg664, while simultaneously favoring outward motion of Leu665 and Pro666. The latter structural change enables the substrate to bind leading to its resultant phosphorylation. Inward motion of Arg664 allows it to interact with the \textgamma-phosphate of ATP as well as the substrate tyrosine. We show that this stabilizes the tyrosine and primes it for the catalytic phosphotransfer, and it may lower the activation barrier of the phosphotransfer reaction. Our work demonstrates the value of including dynamic information gleaned from computer simulation in deciphering RTK regulatory function.\\ \\
This entire abstract was taken from Karb et al. \cite{Karp2017}.
We refer to Figure \ref{fig:f01}
We refer to Formula \ref{useful_name}